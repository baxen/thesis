\chapter{Summary and Outlook}

\label{ch:conclusion}

The search described herein targetted the unique signature of \TeV-scale, charged \acp{LLP}, which are predicted in a variety of extensions to the \ac{SM} including some versions of \ac{SUSY}.
The dataset of 13 \TeV proton-proton collisions was collected during 2015 by the ATLAS detector at the \ac{LHC}, with an integrated luminosity of \lumi~fb\tsup{-1}.
The specific search strategy focused on identifying massive, charged particles which propagate through the Pixel detector in ATLAS by their characteristically large ionization.
Recent updates to the strategy also include a number of rejection techniques that significantly reduce \ac{SM} backgrounds compared to previous iterations.
The analysis also provided a data-driven background estimation method that was shown to be effective with validation tests in both simulation and actual data.

No significant excesses above the background prediction were found in the data, and so limits were placed on the production of massive, charged, \acp{LLP}.
Using a benchmark model of simulated \rhadrons, cross sections above 10-100 fb were excluded at 95\% confidence level, depending on the lifetime of the \rhadron.
Together with the predicted gluino pair-production cross sections, these lead to mass limits on \rhadrons up to 1600 \GeV where the search is most sensitive.
Though these specific values assume an \rhadron \ac{LLP}, the search strategy accomodates a number of other species and the limits can be interpreted for other models.
The search provides a significant contribution to the combined efforts to search for \acp{LLP} in \ac{SUSY}, as the current version places the largest mass limits for lifetimes starting at 3 ns and up through very long lifetimes.

These results are expected to be significantly improved in the following years, primarily because of continuing data collection at 13 \TeV at the \ac{LHC}.
During 2016, but after the release of this analysis, ATLAS recorded an additional 35.5 fb\tsup{-1} of collisions, and analysis of this data would significantly extend the limits presented here.
The next iteration of the analysis can also provide additional interpretations of the search, by explicitly including other models like stop \rhadrons and charginos in the limit calculations, as has been done in previous searches~\cite{SUSY-2014-09}.
This strategy will continue to provide a competitive approach to discovering new \acp{LLP} throughout the lifetime of the \ac{LHC}.

