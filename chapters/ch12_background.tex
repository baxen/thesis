\chapter{Background Estimation}

\label{ch:background}
% --------------------------------------------------------------------------------

The event selection discussed in the previous section focuses on detector signatures, emphasizing a single high-momentum, highly-ionizing track.
That track is then required to be in some way inconsistent with the expected properties of \ac{SM} particles, with various requirements designed to reject jets, hadrons, electrons, and muons (Section~\ref{sec:sm_rejection}.
Were these selections perfectly effective, the signal region would be entirely empty in data.
So the background from this search comes entirely from reducible backgrounds that are outliers of various distributions like momentum, $\dedx$, and \ptcone.
The simulation can be tuned in various ways to do an excellent job of modeling the average properties of each particle type, but it is not necessarily expected to accurately reproduce outliers.
For these reasons, the background estimation used for this search is estimated entirely using data.

% My thoughts on an outline:
%  1) Discuss what can be a background
%  2) Tell the story of dE/dx vs type
%  3) Explain uncorrelated dE/dx and how it can predict
%  4) Caveats about compositon of dE/dx templates
%  6) Show a closure test in MC (no worries about particle type)
%  7) Show validation (and claim t gives some confidence in particle type)

% Here's roughly what the paper does:
% 
%
%
%
%

\section{Background Sources}

% ----------------------------------------

\section{Prediction Method}

% ----------------------------------------

\section{Validation}

% ----------------------------------------
