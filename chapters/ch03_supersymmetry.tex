\chapter{Supersymmetry}

\label{ch:supersymmetry}
% --------------------------------------------------------------------------------

The theory of \ac{SUSY} presents an extension to the \ac{SM} that solves a number of the outstanding issues. 
It is based an another proposed symmetry, one which introduces an equality between the fermionic particles and proposed bosonic partners.
The symmetry is defined by extended spacetime into a superspace, which includes on dimension that describes a particle's spin: a transformation in this spacemoves a fermion with spin-1/2 to a boson with spin-0 or vice-versa.
Requiring the \ac{SM} to be symmetrical under these transformations requires the existence of a bosonic partner for every current matter fermion in the \ac{SM} and a fermionic partner for every boson. 
The partners are called sparticles, where quarks partner with squarks and leptons partner with sleptons, and each boson has a fermionic partner called a gaugino, such as the wino for the W boson.
The superpartners, in the original form of the theory, should be identical to the original particle in every way except for spin; that is they would have the same quantum charges and the same mass.

However, the simplest version of the theory, where the symmetry is unbroken, is incompatible with current observations of physics in a number of systems.
For example, the existence of an electron with spin-0 would introduce a stable, electrically charged constituent of atoms that would not folow the Pauli exclusion principle and would thus significantly change atomic structure.
Various high energy physics measurements have also confirmed the spin of the W and Z bosons, for example, and a fermionic gaugino has never been produced at those masses.
The solution to this incompatibility with observation is to conjecture that the symmetry exists but is broken, where the masses of the supersymmetric particles are significantly larger than those of the current \ac{SM} particles. 

% ----------------------------------------

\section{Structure and the MSSM}

There are a number of ways to model \ac{SUSY}, but many of the resulting phenomena are similar, and a discussion of an example is sufficient describe the structure and results of the theory.
The \ac{MSSM} is one example of a complete description that includes the necessary symmetry breaking to result in the different masses between particles and sparticles.
It is called minimal because it is designed to use the simplest possible extension to the \ac{SM} that incorportaes \ac{SUSY} and remains self consistent.
The theory includes a sparticle partner for every standard model particle, which are listed in Table~\ref{tab:sparticles}.

\begin{table}
\centering
\begin{tabular}{lcc}
\hline
Sector & Particles & Sparticles \\
\hline
Baryonic Matter & $(u,d)$ & $(\tilde{u},\tilde{d})$ \\
                & $(c,s)$ & $(\tilde{c},\tilde{s})$ \\
                & $(t,b)$ & $(\tilde{t},\tilde{b})$ \\
Leptonic Matter & $(\nu_e,e)$ & $(\tilde{\nu_e},\tilde{e})$ \\
                & $(\nu_\mu,\mu)$ & $(\tilde{\nu_mu},\tilde{\mu})$ \\
                & $(\nu_\tau,\tau)$ & $(\tilde{\nu_\tau},\tilde{\tau})$ \\
Higgs           & $(H_u^+, H_u^0)$ & $(\tilde{H}_u^+, \tilde{H}_u^0)$ \\
                & $(H_d^0, H_d^-)$ & $(\tilde{H}_d^0, \tilde{H}_d^-)$ \\
Strong          & $g$ & $\tilde{g}$ \\
Electroweak     & $(W^\pm, W^0)$ & $(\tilde{W}^\pm, \tilde{W}^0)$ \\
                & $B^0$ & $\tilde{B}^0$ \\
\end{tabular}
\caption{The particles in the \ac{SM} and their corresponding superpartners in the \ac{MSSM}.}
\label{tab:sparticles}
\end{table}

To then provide the different masses for those sparticles, the \ac{MSSM} then introduces a second Higgs interaction.
The resulting scalar field, along with the original Higgs field, generates five total particles, $h^0$, the original Higgs boson, $A^0$, $H^0$, and $H^\pm$, where the last two are electrically charged.
These Higgs bosons can mix with the supersymmetric gauginos to form a series off mass eigenstates.
These are usually referred to by the order of their masses, where the neutral gauginos (neutralinos) are labelled $\tilde{\chi}_1^0$, $\tilde{\chi}_2^0$, $\tilde{\chi}_3^0$, and $\tilde{\chi}_4^0$. 
The charged gauginos (charginos) are similarly labelled $\tilde{\chi}_1^\pm$ and $\tilde{\chi}_2^\pm$. 

In addition to the new particle content, the \ac{MSSM} introduces  new interactions for the gauge bosons and gauginos.
All interaction terms are added to the Lagrangian which describe the interaction of a gauge boson or gaugino with a particle or sparticle with the appropriate charge.
Such terms include a few interactions which would violate the observed $B - L$ symmetry that prevents proton decay.
Either the couplings on these terms must be fine-tuned to match the experimental limits on those decays, or an additional symmetry must be imposed to exclude the terms.
The \ac{MSSM} and several other \ac{SUSY} models choose to introduce a new symmetry known as R-parity.
Sparticles are R-parity odd while \ac{SM} particles are R-parity even.
And by requiring that each term in the supersymmetric Lagrangian conserves R-parity, it is enforced that sparticles are produced in pairs.

The conservation of R-parity removes the $B-L$ violating terms naturally from the Lagrangian.
The remaining terms include all of the interactions of the \ac{SM} where two of the particles are replaced with their \ac{SUSY} partners, so that R-parity is conserved in the interactions.
This also has an important significance in making the \ac{LSP}, the $\tilde{\chi}_1^0$, stable, as it cannot decay to only \ac{SM} particles without violating the conservation of R-parity.
The heavier sparticles then decay in chains, emitting an \ac{SM} particle in each step, and leaving behind the \ac{LSP} at the end of the chain.

\section{Motivation}

