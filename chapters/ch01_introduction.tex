\chapter{Introduction}

\label{ch:introduction}

As of 2012, with the discovery of the Higgs boson, the \ac{SM} provides a complete and validated description of the interactions of fundamental particles.
It describes a remarkable range of phenomena given its simple foundation, and has been successful in explaining high energy physics in all experiments yet performed.
However, it is clear that the picture is incomplete: without a description of gravity or an explanation for dark matter, an extension is necessary to describe new physics at higher energies.
These defficiencies motivate a wide range of experiments that search for new physics.
The \ac{LHC} provides the highest energy approach, seeking to discover unobserved particles or interactions in high energy proton collisions.

The experiments at the \ac{LHC} have searched for a wide range of new phenomena in the years since collisions began in 2010.
A major focus of these searches has been on \ac{SUSY}, an extension to the \ac{SM} which has the potential to ameliorate many of its shortfalls.
None of the searches have found evidence of new physics, and between them they have begun to rule out a number of models that would predict new particles at the \TeV scale.
This motivates searches for more exotic signals that may have been missed, using analysis techniques that provide additional reach for the more specific cases.

This dissertation presents a search for \acp{LLP} using the 13 \TeV collisions collected during 2015 at the \ac{LHC}.
Charged \acp{LLP} are predicted to exist in a subset of \ac{SUSY} models, and have dramatically different detector signatures than both \ac{SM} processes and other \ac{SUSY} models.
This search focuses on isolating that unique signature using ionization in the ATLAS detector. 

Part~\ref{part:theory} provides the theoretical context and motivation for a search for new physics in high energy collisions.
Chapter~\ref{ch:standardmodel} outlines the basic framework of the \ac{SM} and describes its particles and interactions.
It also discusses the limitations of the \ac{SM} that motivate the existence of new physics.
Chapter~\ref{ch:supersymmetry} discusses on possible solution to the shortcomings of the \ac{SM}, the theory of \acl{SUSY}, and the ways that it can generate \ac{LLP}.

Part~\ref{part:experiment} discusses the structure of the accelerator complex that provides collisions as well as the experiment that measures them.
Chapter~\ref{ch:lhc} summarizes the design and performance of the \ac{LHC} and the features of the proton-proton collisions it produces.
Chapter~\ref{ch:atlas} then discusses the components of the ATLAS detector and how they can be used to measure the particles produced in \ac{LHC} collisions.
The part concludes with a description of the algorithms used to reconstruct physics particles and processes from the electronic signals in the detector in Chapter~\ref{ch:reconstruction}.

Part~\ref{part:calorimeter} presents a measurement of calorimeter response, an important component of event reconstruction used in many physics analyses.
Chapter~\ref{ch:singlehadrons} describes a direct, in situ measurement of caloriemter response using isolated hadrons, and investigates the modeling of that response in simulation.
Chapter~\ref{ch:jes} uses those measurements to construct a correction for the energy for jets in simulation, the \ac{JES}, and to estimate an uncertainty for that correction.

Part~\ref{part:llp} details the search for \acp{LLP}.
It begins with a discussion of the simulation of \acp{LLP} in ATLAS, focusing on the detector signatures and how they vary with the properties of those particles in Chapter~\ref{ch:simulation}.
Then Chapter~\ref{ch:selection} discusses the strategy of the search and the requirements used to select \acp{LLP} and to reject \ac{SM} backgrounds.
Chapter~\ref{ch:background} explains a method for predicting the background from \ac{SM} processes, and shows a validation of the technique.
Chapter~\ref{ch:systematics} describes the systematic uncertainties on both the selection efficiency for signal events and the background method.
The results of the search are presented in Chapter~\ref{ch:results}. 


